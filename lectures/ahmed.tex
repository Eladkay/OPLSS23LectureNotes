\section{Logical Relations}
\label{sec:logicalrels}
\subauthor{Amal Ahmed}{Northeastern University}
\plquote{``Logical relations turned ASMR.''}

\subsection{Motivation}
Logical relations - a method for writing proofs. This method can generalize for many features of a type system, and has been
very useful in proving a large variety of properties. For example:
\begin{enumerate}
\item \textbf{Termination}. Logical Relations can be used to prove termination properties of programming
languages. For example, we will prove that the simply-typed \lc is terminating.
\item \textbf{Type Soundness/Safety}. Pierce's textbook TAPL\footnote{Types and Programming Languages, chapter 8} discusses
proofs by progress and preservation. This will be discussed further later.
\item \textbf{Program Equivalence}. For example, for verification of subtle algorithms and implementations, or compiler optimizations. 
\item \textbf{Cross-language Transformations}. It is even possible to write cross-language equivalence proofs, especially
useful for cross-language transformations like compilers which essentially translate code between languages and
representations.
\item \textbf{Representation Independence}. The interface should not meaningfully leak details about the implementation
of data structures. This is essentially also an equivalence property.
\item \textbf{Parametricity}. Let's assume our language has no facilities to inspect values' types. Consider
the parametric function type $\forall \alpha. \alpha \to \alpha$. What values inhabit this type? We claim
that only the identity function inhabits this type. This is related to the topic of free theorems, discussed in
Wadler's 1989 paper  ``Theorems for free!''\footnote{\href{https://www2.cs.sfu.ca/CourseCentral/831/burton/Notes/July14/free.pdf}
{https://www2.cs.sfu.ca/CourseCentral/831/burton/Notes/July14/free.pdf}}.
\item \textbf{Security Typing}. Security-typed languages have ``high security'' and ``low security'' types. For example,
we may have $x: \mathbb{N}^\mathrm{H}$ and $y: \mathbb{N}^\mathrm{L}$. We would like to prevent explicit flow
of information from high-security values to low-security observers, such as $y = x$. In addition, we'll also have to
prevent implicit flow, like $y = x > 0 ? 1 : 0$. How can we prove that the type system does, in fact, preserve
confidentiality? How would we even phrase that as a theorem?
\end{enumerate}

\subsection{Introduction to Logical Relations}
We consider \emph{logical predicates} which are unary and \emph{logical relations} which are binary. 
Logical predicates are denoted $P_\tau (e)$, where $\tau$ is a type and $e$ is a program. For example, termination and type
soundness are unary properties. 
Logical relations are denoted $P_\tau (e_1, e_2)$, where $\tau$ is a type and $e_1, e_2$ are programs. For example, program equivalence
and representation independence are binary properties. This $\tau$ is usually taken as the ``source'' type, and there would
also be a destination type.

We will be proving the termination of all well-typed programs in the simply-typed \lc. This is also called \emph{normalization}.

Consider the following variation on the simply-typed \lc:
\[
\begin{array}{rl}
\tau ::= &\mathrm{bool} \mid \tau_1 \to \tau_2 \\
e ::= &x \mid \mathrm{true} \mid \mathrm{false} \mid \mathrm{if}~e~\mathrm{then}~e_1~\mathrm{else}~e_2  \\
& \mid \lambda x: \tau. e \mid (e e) \\
v ::= & \mathrm{true} \mid \mathrm{false} \mid \lambda x: \tau. e \\
E ::= & [ \cdot ] \mid \mathrm{if}~E~\mathrm{then}~e_1~\mathrm{else}~e_2 \mid E e_2 \mid v E \\
& \\
\mathrm{if~true~then}~e_1~\mathrm{else}~e_2 \mapsto e_1 &\\
\mathrm{if~false~then}~e_1~\mathrm{else}~e_2 \mapsto e_2 &\\
(\lambda x: \tau. e) v \mapsto e[v/x]
\end{array}
\]

The rest of the typing and deduction rules are what we would expect.
This is taken to be a call-by-value language and parameters must be evaluated (left to right) before function calls.

We say that $e \Downarrow$ if there exists a value to which $e$ evaluates. Then, we wish to prove $\cdot \vdash e : \tau \to e \Downarrow$.
We may try to prove this by induction: the base cases would be $\mathrm{true}$ and $\mathrm{false}$, and variables. These clearly all
terminate. The inductive cases:
\begin{itemize}
\item Assume $\cdot \vdash e : \mathrm{bool},~\cdot\vdash e_1 : \tau,~\cdot\vdash e_2 : \tau$, and by the IH, their evaluation 
terminates. Assume that $e \to^* v,~e_1\to^* v_1,~e_2 \to^* v_2$. We wish to conclude that the evaluation of the well-typed expression
$\mathrm{if}~e~\mathrm{then}~e_1~\mathrm{else}~e_2$ terminates. This expression reduces to $\mathrm{if}~v~\mathrm{then}~e_1~\mathrm{else}~e_2$.
Since $v$ is either $\mathrm{true}$ or $\mathrm{false}$, it terminates.
\item As for the lambda case, it clearly terminates as it is already a value and it is non-reducable.
\item Assume $\cdot \vdash e_1 : \tau_2 \to \tau_1$ and $\cdot \vdash e_2 : \tau_2$, and by the IH, their evaluation
terminates. Assume that $e_1 \to^* \lambda x: \tau_2. e'$ and $e_2 \to^* v_2$. We wish to conclude that the evaluation of the
well-typed expression $e_1 e_2$ terminates. This reduces to $e_1 e_2 \to^* (\lambda x: \tau_2. e') v_2 \to e'[v_2/x]$. 
Now we're stuck, because we don't know anything about the bodies of lambda expressions. Our IH is not strong enough to prove this.
\end{itemize}

\subsection{Proof by Logical Relations}
We will define a \emph{logical predicate} which will assist in this proof. We will define the predicate $N_\tau(e)$
by induction on the type $\tau$:
\[
\begin{array}{rl}
N_\mathrm{bool}(e)\leftrightarrow& \vdash e: \mathrm{bool} \wedge e \Downarrow \\
N_{\tau_1 \to \tau_2}(e)\leftrightarrow& \vdash e: \tau_1 \to \tau_2 \wedge e \Downarrow\\
& \wedge (\forall e' N_{\tau_1}(e') \to N_{\tau_2} (e e'))
\end{array}
\]

We wish our predicate to have three main properties:
\begin{enumerate}
\item $\cdot \vdash e: \tau \to P_\tau(e)$
\item $P_\tau(e) \to$ the condition we want to prove
\item $P_\tau(e)$ should be preserved by the \textbf{elimination forms} for the type $\tau$.
\end{enumerate}
What the third condition means is that the predicate needs to be preserved by usages of that type. 
For example, function application or usage of a Boolean condition. Proving these properties will
clearly imply what we want to prove, but it is a strengthening.

Proving the second property is trivial here, by the definition. Now, let $\gamma$ be a set of substitutions
such that $\gamma \implies Gamma$: we will take this to mean
$\mathrm{dom}(\gamma) = \mathrm(dom){\Gamma}$ and $\forall x\in \mathrm{dom}(\Gamma) N_\Gamma(x) (\gamma(x))$.
$\gamma$ is a mapping from variable names to lambda expressions.
For the first property, we shall prove that if $\Gamma \vdash e : \tau$ and $\gamma$ is a set of substitutions
such that $\gamma \models \Gamma$, then $N_\tau(\gamma(e))$ for all $e$. That means that $N_\tau$ is preserved
under substitution.

We will use the following substitution lemma: If $\Gamma \vdash e : \tau$ and $\gamma \models \Gamma$, 
then $\vdash \gamma(e) : \tau$. Also, we will use the following lemma: If $\cdot \vdash e : \tau$ and
$e \to e'$ then $N_\tau (e) \leftrightarrow N_\tau (e')$ (forward and backward reduction preservation).

Now, we will prove the following theorem which will imply the first property: 
$\Gamma \vdash e : \tau \wedge \gamma \models \Gamma \to N_\tau (\gamma(e))$. 
We will prove this by induction on the typing rules.

For the case $\frac{}{\Gamma \vdash \mathrm{true} : \mathrm{bool}}$, we have $\gamma \models \Gamma$
and want to show $N_\mathrm{bool} (\gamma(\mathrm{true}))$. This is equivalent to $N_\mathrm{bool} (\mathrm{true})$
This does hold, and so we are done.

For the case $\frac{\Gamma \vdash e_1 : \tau_2 \to \tau~~~~\Gamma \models e_2 : \tau_2}{\Gamma \vdash e_1 e_2 : \tau}$, 
we have $\gamma \models \Gamma$ and want to show $N_\tau (\gamma(e_1 e_2))$. 
This is equivalent to $N_\tau (\gamma(e_1) \gamma(e_2))$.
By the induction hypothesis, $N_{\tau_2 \to \tau}(\gamma(e_1))$ and $N_{\tau_2}(\gamma(e_2))$.
We can use the $\forall$ part of the definition of $N_{\tau_2 \to \tau}(\gamma(e'))$ with $e' = \gamma(e_2)$. 
We know that $N_{\tau_2} (e')$ and so $N_\tau (\gamma(e_1) \gamma(e_2))$ which is what we wanted to show.

For the case $\frac{\Gamma, x : \tau_1 \vdash e : \tau_2}{\Gamma \vdash \lambda x: \tau_1. e: \tau_1 \to \tau_2}$,
we have $\gamma \models \Gamma$ and we want to show $N_{\tau_1 \to \tau_2} (\gamma(\lambda x: \tau, e))$.
This is equivalent to $N_{\tau_1 \to \tau_2}(\lambda x: \tau_1 \gamma(e))$.
We will do this by definition of $N_{\tau_1 \to \tau_2}$.
\begin{enumerate}
\item $\vdash e: \tau_1 \to \tau_2$ by our assumption and the definition of $\lambda$.
\item $\lambda x: \tau. \gamma(e) \Downarrow$ because it's just a value.
\item Let $e'$ such that $N_{\tau_1}(e')$. Show that $N_{\tau_2}((\lambda x: \tau_1. \gamma(e)) e')$.
By the induction hypothesis, if $\gamma' \models \Gamma, x: \tau_1$, then $N_{\tau_2}(\gamma'(e))$.
We will pick $\gamma' = \gamma[x\mapsto v']$ such that $N_{\tau_1} (v')$. This $v'$ is the 
reduced form of $e'$. By forward-reduction preservation and $N_{\tau_1}(e')$, we can conclude 
$N_{\tau_1}(v')$. We can conclude $N_{\tau_2}(\gamma'(e))$, which is $N_{\tau_2}(\gamma(e)[v'/x])$,
which is $N_{\tau_2}((\lambda x: \tau_1. \gamma(e)) e')$ which is what we wanted to prove.
\end{enumerate}
\qed

\subsection{Type Safety}
We will prove the following theorem:

\textbf{Theorem. (Type Safety)} If $\cdot \vdash e: \tau$ and $e\to^* e'$, then either
$e'$ is a value or $\exists e''. e' \to e''$. \footnote{An alternative perspective on type safety: 
\href{https://doi.org/10.1145/3236766}{https://doi.org/10.1145/3236766}}

This means we never get stuck. This can be proven by the method of progress and preservation:

\begin{itemize}
\item \textbf{Progress.} If $\cdot \vdash e: \tau$ then either $e$ is a value or $\exists e''. e \to e'$.
\item \textbf{Preservation.} If $\cdot \vdash e : \tau$ and $e \to e'$ then $\cdot \vdash e' : \tau$.
\end{itemize}

Preservation is typically proven by induction on the evaluation steps. We will prove type safety
by using logical relations. % We will define the unary predicate $P_\tau (e)$.

First, we will define a type-value set mapping and a type-expression set mapping:
\[
\begin{array}{c}
\mathcal{V}\llb \mathrm{bool} \rrb = \{\mathrm{true, false}\} \\
\mathcal{V}\llb \tau_1 \to \tau_2 \rrb = 
\{\lambda x: \tau_1. e \mid \forall v \in \mathcal{V}\llb \tau_1 \rrb. e[v/x] \in \mathcal{E}\llb \tau_2 \rrb\} \\ 
\\
\mathcal{E}\llb \tau \rrb = \{e \mid \forall e'. (e \to^* e' \wedge \not\exists e''. e' \to e'') \to e' \in \mathcal{V}\llb \tau \rrb\}
\end{array}
\]

The values are either booleans, or functions such that applying the function with an argument of the correct type
gives an expression fo the correct type. The expressions are all those that when reduced completely give a value
of the correct type.

We would like to define a logical relation that has the following properties, and thus get what we want: 
\begin{itemize}
\item $\cdot \vdash e: \tau \to e \in \mathcal{E}\llb \tau \rrb$
\item $e\in \mathcal{E} \llb \tau \rrb \to e$ is safe, i.e., making progress gets us to either a 
value or something that can be reduced further.
\end{itemize}

First, we will define the ``interpretation'' of a typing environment:
\[
\begin{array}{rl}
\mathcal{G}\llb \cdot \rrb = & \{\emptyset\} \\
\mathcal{G}\llb \Gamma, x : \tau \rrb = & 
\{\gamma[x\mapsto v] \mid \gamma \in \mathcal{G}\llb \Gamma \rrb \wedge v\in \mathcal{V}\llb \tau \rrb\}
\end{array}
\]
This is the set of assignments consistent with the type environment.

Now, we define \emph{semantic} type safety:
\[
\Gamma \models e: \tau \leftrightarrow \forall \gamma \in \mathcal{G}\llb \Gamma \rrb. \gamma(e) \in \mathcal{E}\llb \tau \rrb
\]

We shall now prove the \textbf{fundemental property} or \textbf{basic lemma} of our logical relation: If 
$\Gamma \vdash e : \tau$ then $\Gamma \models e : \tau$.

We will prove one case of the induction needed to prove this property:
$\frac{\Gamma, x : \tau_1 \vdash e : \tau_2}{\Gamma \vdash \lambda x: \tau_1. e : \tau_1 \to \tau_2}$
We need to show $\Gamma \models \lambda x: \tau_1. e : \tau_1 \to \tau_2$. Let $\gamma \in \mathcal{G}\llb \Gamma \rrb$.
We need to show that $\gamma (\lambda x: \tau_1. e) \in \mathcal{E} \llb \tau_1 \to \tau_2 \rrb$. This is equivalent to
$\lambda x: \tau_1. \gamma(e) \in \mathcal{E} \llb \tau_1 \to \tau_2 \rrb$.

So, suppose $\lambda x: \tau_1. \gamma(e) \to^* e'$ and $e'$ is irreducible. Actually, since $\lambda x: \tau_1. \gamma(e)$ is
a value, it is $e'$ with zero derivation steps. 
We need to show that $\lambda x: \tau_1. \gamma(e) \in \mathcal{V}\llb \tau_1 \to \tau_2 \rrb$.
Suppose that $v\in \mathcal{V}\llb \tau_1 \rrb$. We need to show $\gamma(e)[v/x] \in \mathcal{E} \llb \tau_2 \rrb$.

By the IH, we know that $\Gamma, x : \tau \models e : \tau_2$. Instantiate the definition with $\gamma' = \gamma[x\mapsto v]$.
Then, we get $\gamma[x\mapsto v]\in\mathcal{E} \llb \tau_2 \rrb$ and so $\gamma(e[v/x]) \in \mathcal{E} \llb \tau_2 \rrb$.
And so, we are done.

\qed

\subsection{Recursive Types}

We will define a new type constructor: tagged unions.

\[
\begin{array}{rl}
\mathcal{V}\llb \tau_1 + \tau_2 \rrb = & \{\text{inl~} v_1 \mid v_1 \in \mathcal{V} \llb \tau_1 \rrb\} \\
& \cup \{\text{inr~} v_2 \mid v_2 \in \mathcal{V} \llb \tau_2 \rrb\}

\end{array}
\]


Consider the following type in Coq:
\begin{lstlisting}
Inductive tree : Set =
| Leaf : tree
| Node : nat -> tree -> tree.
\end{lstlisting}

This can be written by using a recursive equation: $\alpha = 1 + \mathbb{N}\times \alpha\times \alpha$.
This expands to $\alpha = 1 + \mathbb{N}\times (1 + \mathbb{N}\times \alpha\times \alpha)\times (1 + \mathbb{N}\times \alpha\times \alpha)$,
and so on.

Essentially, if we consider $F = \lambda x: \star. 1 + \mathbb{N}\times \alpha\times \alpha$, then we're looking for
a fixed point of $F$.

We will define a fixed-point type constructor: $\mu \alpha. F(\alpha)$. We will take this constructor to have
the property: $\mu \alpha. F(\alpha) = F(\mu \alpha. F(\alpha))$.

In other words, if we let $\tau = F(\alpha)$: $\mu \alpha. \tau = F(\mu \alpha. \tau) = \tau[\mu \alpha. \tau / \alpha]$/
Moving to the right in this equation is also called ``unfolding'' and moving to the left is ``folding''.

We will extend our simply-typed \lc and add $\tau ::= \mu \alpha. \tau$ as a type derivation.
Also, add $e ::= \mathrm{fold} e \mid \mathrm{unfold} e$ to the expressions and $v ::= \mathrm{fold} v$
to the values. The operational semantics are $\mathrm{unfold} (\mathrm{fold} v) \to v$.
\todo{Add typing rules}

Now, we can define the type of integer lists as $\mu \alpha. 1 + (\mathbb{N} \times \alpha)$.

We can type-check self application: $\mathrm{SA} = \lambda x: (\mu \alpha. \alpha \to \tau). (\mathrm{unfold} x) x$.
Now, we can type-check $\Omega: \tau = \mathrm{SA} (\mathrm{fold SA})$. Wait, what's $\tau$? It's completely
arbitrary. This means that $\Omega$ evaluates to a value of any type $\tau$. It's fine because it never finishes
evaluation.

We would like to modify our logical relation to prove type safety. 
We may try to define 
$\mathcal{V} \llb \mu \alpha. \tau \rrb = \{ \mathrm{fold} v \mid v \in \mathcal{V}\llb \tau[\mu \alpha. \tau / \alpha] \rrb\}$.
However, this is not well founded. 

This problem has remained open for a while. 
In the second edition of ATTAPL\footnote{\href{https://www.cis.upenn.edu/~bcpierce/attapl/}{https://www.cis.upenn.edu/~bcpierce/attapl/}}, 
it was given as an ``exercise'' with infinitely many difficulty stars. The solution
is using ``step index logical relations''. Essentially, this is a stratification of $\mathcal{V}$.
What $\mathcal{V}_k$ will be is ``values that are indistinguishable from corresponding values from $\mathcal{V}$ using up to
$k$ steps''.

The Boolean case is easy because it does not change: $\mathcal{V}_k\llb \mathrm{bool} \rrb = \{\mathrm{true}, \mathrm{false}\}$.
\[
\begin{array}{c}
\mathcal{V}_k\llb \tau_1 \to \tau_2 \rrb = \{\lambda x: \tau_1. e \mid \forall j \leq k. \forall v\in \mathcal{V}_j\llb\tau_1 \rrb.
e[v/x] \in \mathcal{E}_j \llb \tau_2 \rrb\} \\
\mathcal{V}_k\llb\mu \alpha. \tau \rrb = \{\mathrm{fold~} v \mid \forall j < k. v \in 
\mathcal{V}_j\llb \tau[\mu \alpha. \tau / \alpha \rrb\} \\
\mathcal{E}_k \llb \tau \rrb = \{e \mid \forall j < k. \forall e'. e \to^j e' \wedge e' 
\mathrm{~is~irreducable} \to e' \in \mathcal{V}_{k - j}\llb \tau \rrb\} \\
\mathcal{G}_k \llb \cdot \rrb = \{\cdot \}
\mathcal{G}_k \llb \Gamma, x: \tau \rrb = \{\gamma[x\mapsto v] \mid \gamma \in \mathcal{G}_k \llb \Gamma \rrb 
\wedge v\in \mathcal{V}_k\llb \tau \rrb\} \\
\Gamma \models e : \tau \equiv \forall k \geq 0. \forall \gamma \in \mathcal{G}_k \llb \Gamma 
\rrb. \gamma(e) \in \mathcal{E}_k \llb \tau \rrb
\end{array}
\]

\textbf{Theorem. (Fundemantal Property)} $\Gamma \vdash e : \tau \to \Gamma \models e : \tau$

The stratification is needed in cases where we have circularities. Recursive types are one such case.
Another case is when we have mutable references: these references may have circular dependencies.
Non-circular operations can count as ``zero-step'' operations and not change the level we are on.
In this case, the only ``one-step'' operation we need is $\mathrm{fold}$.

