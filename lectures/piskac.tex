\section{Program Synthesis}
\subauthor{Ruzica Piskac}{Yale University}
\plquote{``Of course LTL is decidable! It's double-exponential complete.''}
\subsection{Overview}

The idea: Generally, the way we implement systems is by writing a specification, and then
writing an implementation, and then verify both the specification and the implementation.
Program synthesis works by constructing \emph{correct} programs automatically
from the specification.

This allows us to write less code and get better results, and also have provably
correct programs.

Church synthesis problem (1957): Given a requirement $\varphi$ on the
input-output behavior of a Boolean circuit, compute a circuit $C$
satisfying the condition.

Reactive synthesis: Given an LTL specification, automatically compute the progrm
that satisfies the specification.

The basic definition of a synthesis problem: Given a specification $\varphi$,
return a program that satisfies the specification, or declare that one does not exist.

\subsection{LTL}

LTL (Linear Temporal Logic) concerns itself with infinite paths in a state system. 
Syntax and semantics of LTL path formulas:
\[
\begin{array}{cc}
\text{LTL}&\text{Meaning} \\
p & p \\
X \varphi & \text{in next state it holds that~} \varphi \\
G \varphi & \text{in all following states it holds that~} \varphi \\
F \varphi & \text{in some following state it holds that~} \varphi \\
\varphi U \psi & \text{in some following state it holds that~} \psi \text{~and until then, it holds that~} \varphi
\end{array}
\]

In addition, we can express $A\varphi$, where $\varphi$ is an LTL path formula, which means 
``all paths out of a state satisfy $\varphi$''.

We consider two types of questions:
\begin{itemize}
\item \textbf{Satisfiability:} Is there a \emph{trace} satisfying the spec?
\item \textbf{Realizability:} Is there a \emph{system} satisfying the spec?
\end{itemize}

For example, consider the following LTL specification $G ((p_1 \to q_1) \wedge (p_2 \to q_2) \wedge \neg (q_1 \wedge q_2))$
where $p_1, p_2$ are inputs and $q_1, q_2$ are outputs.
This is satisfiable, for example by the trace that never satisfies $p_1$ or $p_2$.
However, it is not realizable because we do not control $p_1$ or $p_2$: they may
be both on at the same time.

Another example: $G (p \leftrightarrow Xq)$. This is obviously satisfiable, but it 
is not realizable because the system cannot look into the future.

\subsection{Reactive Synthesis}
Synthesis can be seen as a game: the verifier proves inputs and the prover provides outputs.
The system is good if it fulfills the spec for all possible inputs.

In some cases, the synthesis problem is easy to formulate but hard to solve:
for example, consider the synthesis problem in which the inputs are
black moves in a game of Chess, and the outputs are white moves. The
system would be considered good if whenever the black moves
are legal, the white moves are legal as well and white reaches
checkmate.

We define the model of \emph{deterministic B\"uchi automaton}: 
an automaton on infinite words that accepts an infinite word
if it passes through an accepting state infinitely many times.

By constructing the B\"uchi automaton for an LTL formula,
we can create a model that can then be translated to code.

This is the method of reactive synthesis: we give a specification,
create a game (which is represented by the B\"uchi automaton), 
solve it and create a system.

Formally, a game is a 4-tuple $A = (V_0, V_1, E, F)$ such that $V_i$ is the $i$\textsuperscript{th}'s player's states,
$E \subseteq (V_0 \times V_1) \cup (V_1 \times V_0)$ is the set of allowed transitions and $F$ is the 
winning condition. A run is an infinite set of states
connected by edges from $E$.

Winning conditions:
\begin{itemize}
\item \textbf{Reachability}: Player 1 wins if the run passes through a state in some set $F$.
\item \textbf{B\"uchi}: Player 1 wins if the run passes infinitely many times through a state in some set $F$.
\item \textbf{Parity}: Let $C: V_0 \cup V_1 \to \N$ be a coloring function. 
Player 1 wins if the color with the highest number that is seen, is seen an even number of times.
\end{itemize}

Now, we can translate an LTL formula to a nondeterministic B\"uchi automaton, and then determinize it
into a deterministic \emph{parity} automaton. Then, rephrase as a parity game and solve.

We have to go through this determinization process because NBAs may have false negatives due to
the multiple possible runs for a set of inputs. However, there exist LTL formulas with no
equivalent DBA.

\subsection{Deductive Synthesis}
Initially, computers and tools were too weak for synthesis. In 1969, Cordell Green tried
to synthesize a sorting algorithm using a then-state of the art theorem solver.
It did not work, but the work contained a full method of extracting a program from a resolution
proof.

Today, most synthesis tools are based on SMT solvers. Artificial intelligence tools also play an important part.
Synthesis can be used to generate parts of the code, or entire algorithms. SMT solvers are
tools used to prove arbitrary theorems and try to decide arbitrary formulas. SMT solvers combine
propositional solving techniques with various \emph{theories}.

For example, consider having some \lstinline{big_set} variable in a program, and seeking
two \emph{disjoint} sets \lstinline{set_a, set_b} with the same size such that their union
is the big set. Solving this logical constraint to get actual code corresponds to constructively solving
the quantifier elimination problem $\exists x. F(x, a)$.

We define the \textbf{synthesis procedure} as one taking as input formula $F(x, a)$ and having
as outputs:
\begin{itemize}
\item A precondition formula $\mathrm{pre}(a)$
\item A list of terms $\Psi$
\end{itemize}
Such that the following holds: $\exists x. F(x, a) \leftrightarrow \mathrm{pre}(a) \leftrightarrow F[x/\Psi]$.

The synthesis procedure for the linear integer arithmetic can be implemented by
inlining equality constraints and passing inequality constraints to an SMT solver.
This is especially useful due to arithmetization: it is possible to reduce
many problems to integer arithmetic problems.

\subsection{Syntax-Guided Synthesis}

\textsc{Sketch} is a synthesis method that was pioneered by Armando Solar-Lezama when he was a PhD student.
The basic idea is giving a partial program with \emph{holes}, and then \textsc{Sketch} will look through all
possible ways to fill the hole such that the specification is satisfied.
This opened the door for additional SyGuS paradigms: Syntax-Guided Synthesis. These are
synthesis methods that are based on the syntax on the synthesized program.

One possible mode of specification is PBE: Programming by Example. This mode specifies the synthesis problem by giving a few
examples of input-output pairs. For example, \textsc{FlashFill} in Microsoft Excel\textsc{TM} is a PBE synthesizer.

A use for PBE and SyGuS is superoptimization: try to synthesize the shortest or most efficient program that
is equivalent to a given problem. Another use is invariant generation: trying to find loop invariants
for given loops.

Formally, SyGuS is defined as finding a program snippet $e$ such that:
\begin{itemize}
\item $e\in E$ (syntactic constraint)
\item $e \models \varphi$ (semantic contraint)
\end{itemize}
This is a core computational problem in many synthesis tools and applications.
Like SMT problems can be phrased in the SMT-LIB input format, there is SYNTH-LIB
for synthesis problems, which can be read by tools such as CVC5.

One possible method for solving synthesis problems is CEGIS: counterexample-guided inductive synthesis.
More methods are enumerative (by exhaustive search), symbolic (using constraint solving) and
stochastic (a probabilistic walk).