\section{Introduction to Category Theory}
\subauthor{Daniela Petrisan}{Institut de Recherche en Informatique Fondamentale}
\plquote{``It's morphisms all the way down.''}
\subsection{Overview}
Category theory links many topics together. It involves a few fundemental concepts:
\begin{enumerate}
\item Categories.
\item Functors.
\item Natural transformations.
\item Limits and colimits.
\item Adjunctions
\item Monads
\end{enumerate}
Ok, maybe more than a few. This course will demonstrate category theory using examples from
the field of \emph{automata theory}, like minimization and learning algorithms, as well
as the field of \emph{computational effects}. Category theory involves objects and structures,
and relations between them. 
For example, the objects can be sets, denoted $A, B, C$, and the relations can be functions:
\begin{figure}
\begin{tikzpicture}[->,>=stealth',shorten >=1pt,auto,node distance=2.1cm,scale = 1,transform shape]
\tikzset{pathsnake/.style={decorate, decoration={snake}, draw=black}}; 
\node      (A)                              {A}; 
\node      (B)      [right of=A]            {B}; 
\node      (C)      [below of=B]            {C}; 
\node      (D)      [right of=C]            {D}; 
\draw      (A)   	node {$f$}				(B);
\draw      (B)   	node {$g$}				(C);
\draw      (A)   	node {$g\circ f$}		(C);
\draw      (C)   	node {$h$}				(D);
\draw      (B)   	node {$h\circ g$}		(D);
\end{tikzpicture}
\end{figure}
\todo{make this work}

\subsection{Introduction to Categories}
\textbf{Definition. (Category)} A category $C$ consists of the following data:
\begin{itemize}
\item A class of objects $A, B, ...$
\item A class of arrows (morphisms) $f, g, h, ...$. 
For each $f$ we have $\mathrm{dom}(f)$ and $\mathrm{codom}(f)$, and we write
$f: A \to B$ if $A = \mathrm{dom}(f)$ and $B = \mathrm{codom}(f)$.
\item A partial composition: If $f: A \to B$ and $g: B \to C$, we have 
$g \circ f = A \to C$.
For each $A \in C$ we have an identity arrow: $1_A : A \to A$.
% The associativity of composition and unit axioms hold: $(f \circ g) \circ h \equiv f \circ (g \circ h)$ and $f \circ 1_A \equiv f$.
\end{itemize}

We will denote the collection of morphisms from $A$ to $B$: $C(A, B)$ or $\mathrm{hom}(A, B)$.

\textbf{Examples:}
\begin{center}
\begin{tabular}{ |p{2cm}|p{2cm}|p{3cm}|p{3cm}|p{3cm}| } 
 \hline
 Category & Objects & Arrows & Composition & Identity\\ 
 \hline
 \hline
 \textsc{Set} & Sets & Functions & Function composition & Identity function\\ 
 \hline
 \textsc{Rel} & Sets & Relations & Relation composition & Empty relation\\
 \hline
 $(X, \leq)$ & $x\in X$ & $x \to y \leftrightarrow x \leq y $ & Transitivity & For each $x\in X$, $(x, x)$ \\
 \hline
 $(M, \cdot, 1_M)$, a monoid & ${X}$ & $m \in M$ & $\cdot$ & $1_M$ \\
 \hline
 \textsc{Mon} & Monoids & Monoid homomorphisms & Homomorphism composition & Identity function \\
 \hline
 \textsc{Top} & Topological spaces & Continuous maps & Map composition & Identity map \\
 \hline
 \textsubscript{\textsc{Vec}}{$\mathbb{K}$} & Vector spaces & Linear transformations & Transformation composition & Identity transformation \\
 \hline
 \textsc{Pre} & Preorders & Monotone maps & Map composition & Identity map \\
 Free category on a graph $C(G)$ & $G$'s vertices &Paths from $G$, including the empty path& Path concatenation & Empty path \\
 \hline
\end{tabular}
\end{center}

\textbf{Example.} Fix a finite set $A$ and consider the graph:
\begin{figure}
\begin{tikzpicture}[->,>=stealth',shorten >=1pt,auto,node distance=2.1cm,scale = 1,transform shape]
\tikzset{pathsnake/.style={decorate, decoration={snake}, draw=black}}; 
\node      (in)                               {in}; 
\node      (st)      [right of=in]            {st}; 
\node      (out)     [right of=st]           {out}; 
\draw      (in)   	node {$\rhd$}	  (st);
\draw      (st)   	node {$a\in A$}			  (st);
\draw      (st)   	node {$\bowtie$}		  (out);
\end{tikzpicture}
\end{figure}
\todo{make this work}

Then if $J$ is the free category on this graph, then $J(\mathrm{in, st}) = \{\rhd w \mid w\in A^*\}$.
Actually, $J(\mathrm{st, st}) \cong A^*$.


\subsection{Functors}
Let $C, D$ be two categories. A \textbf{functor} $F: C \to D$ associates: 
\begin{enumerate}
\item To each object $A$ in $C$ an object $FA$ in $D$.
\item To each morphism $f: A \to B$ in $C$, a morphism $Ff : FA \to FB$ such that $F (f \circ g) = F(f) \circ F(g)$ and $F(1_A) = 1_FA$.
\end{enumerate}

\textbf{Examples:}
\begin{itemize}
\item \textbf{Forgetful functor} $U : \mathsc{Mon} \to \mathsc{Set}$ by $(M, \cdot, 1_M) \mapsto M$ and 
$f: (M, \cdot, 1_M) \to (N, \cdot, 1_N) \mapsto Uf: M \to N$
\item \textbf{Free construction functor} $F : \mathsc{Set} \to \mathsc{Mon}$ by $X \mapsto (X^*, \cdot, \varepsilon)$
and $f: A \to B \mapsto Ff: (A^*, \cdot, \varepsilon) \to (B^*, \cdot, \varepsilon)$
\end{itemize}
\todo{add more examples}

\subsection{Automata as Categories}

Cosnider the following category, which represents a deterministic automaton over the alphabet $A$:
\begin{figure}
\begin{tikzpicture}[->,>=stealth',shorten >=1pt,auto,node distance=2.1cm,scale = 1,transform shape]
\tikzset{pathsnake/.style={decorate, decoration={snake}, draw=black}}; 
\node      (1)                               {1}; 
\node      (Q)      [right of=in]            {$Q$}; 
\node      (2)     [right of=st]             {2}; 
\draw      (1)   	node {$i$}	  			 (2);
\draw      (Q)   	node {$a\in A$}			 (Q);
\draw      (Q)   	node {$f$}		  		 (2);
\end{tikzpicture}
\end{figure}
\todo{make this work}

Where $Q$ is some set, $1 = \{\star\}$, $2 = \{\top, \bottom\}$, $i(\cdot)$ is the initial state of the automaton and $f$ is the
characteristic function of the set of accepting states,
\todo{add nfa, wfa, and examples}
